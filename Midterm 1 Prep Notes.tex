\documentclass[11pt]{article}

%----------Packages----------
\usepackage{color}
\usepackage{amsmath}
\usepackage{amssymb}
\usepackage{amsthm}
\usepackage{amsrefs}
\usepackage{mathrsfs}
\usepackage{mathtools}
\usepackage[mathcal]{eucal}
\usepackage{enumerate}
\usepackage[shortlabels]{enumitem}
\usepackage{verbatim}
\usepackage{fullpage}
\usepackage{multicol}

%----------Commands----------
\clubpenalty=9999
\widowpenalty=9999

\newcommand{\bbF}{\mathbb{F}}
\newcommand{\bbN}{\mathbb{N}}
\newcommand{\bbQ}{\mathbb{Q}}
\newcommand{\bbR}{\mathbb{R}}
\newcommand{\bbZ}{\mathbb{Z}}

\renewcommand{\phi}{\varphi}
\renewcommand{\emptyset}{\O}

\renewcommand{\_}[1]{\underline{ #1 }}
\DeclarePairedDelimiter{\abs}{\lvert}{\rvert}
\DeclarePairedDelimiter{\norm}{\lVert}{\rVert}

\newcommand{\arr}{\longrightarrow}

\DeclareMathOperator{\ext}{ext}
\DeclareMathOperator{\bridge}{bridge}

%---------- Proof QED change ----------
\renewcommand{\qedsymbol}{QED}

%----------Theorems----------
\newtheorem{theorem}{Theorem}[section]
\newtheorem*{theorem*}{Theorem}
\newtheorem{proposition}[theorem]{Proposition}
\newtheorem{lemma}[theorem]{Lemma}
\newtheorem{corollary}[theorem]{Corollary}

\theoremstyle{definition}
\newtheorem{definition}[theorem]{Definition}
\newtheorem{remark}[theorem]{Remark}
\newtheorem{question}[theorem]{Question}

\numberwithin{equation}{subsection}

%----------Document----------
\begin{document}

\begin{center}
    {\LARGE\bfseries MATH 20700 Honors Analysis I}\\[6pt]
    {\Large Midterm Prep Notes}\\[6pt]
    {\Large\itshape Malhar Manek}
\end{center}

\bigskip

\section*{Counterexamples}
\begin{theorem}
    Infinite union of closed sets that is not closed. Infinite intersection of open sets that is not open. 
    \begin{proof}
        $A_n = [0, 1 - 1/n]$, each closed, but $\bigcup_n A_n = [0,1)$ is not closed.

        $A_n = (-1/n, 1/n)$, each open, but $\bigcap_n A_n = \{0\}$ is not open.
    \end{proof}
\end{theorem}

\begin{theorem}
    A metric space that is compact when defined with one metric, but not one another metric.
    \begin{proof}
        Closed unit ball in infinite dimensions. 

        On the sup norm metric it is not compact.

        On the Euclidean/$\ell_2$ norm metric it is not compact.

        But on the $\sum_{n=1}^\infty \frac{(a_n-b_n)}{2^n}$ metric it is compact.
    \end{proof}
\end{theorem}

\begin{theorem}
    Examples of sets $X,Y$ such that: Interior of closure of $X$ is not $X$. Closure of interior of $Y$ is not $Y$.
    \begin{proof}
        $X=\bbQ$, then $\overline{X}=\bbR$ but $\text{Interior} (\bbR) = \bbR$.

        $Y = \bbQ$, then $\text{Interior}(Y)=\emptyset$ and $\overline{(\text{Interior}(Y))}=\emptyset$.
    \end{proof}
\end{theorem}

\begin{theorem}
    Non-trivial clopen set in a disconnected metric space.
\begin{proof}
Let the metric space be $X = [0,1]\cup [2,3] \subset \bbR$. Then we claim that $[0,1]$ is clopen in $X$.

Closed: We want to show that $[2,3]$ is open in $X$. $[0,1] = (1.5,3.5) \cap X$ so it is open in $X$.

Open: $[0,1] = (-0.5, 1.5) \cap X$ so it is open in $X$. 
\end{proof}
\end{theorem}


\begin{theorem}
Closed and bounded set in a general metric space $M$ that is not compact.
\begin{proof}
Closed unit ball in infinite dimensions equipped with sup norm metric.

Consider the sequence of sequences $(e_n)$ such that the sequence $e_i$ has $1$ in the $i$-th position and 0 everywhere else. Distance between any two sequences is $1$, so there is no Cauchy subsequence hence there is no convergent subsequence.

\end{proof}
\end{theorem}

\begin{theorem}
    Cauchy sequence that does not converge in a disconnected metric space.
\begin{proof}
    Decimal expansion of $\sqrt2$ in $\bbQ$ i.e. the sequence of rationals $1, 1.4, 1.41, 1.414, 1.4142, 1.41423\dots$
\end{proof}
\end{theorem}


\begin{theorem}
Open map that is not continuous.
\begin{proof}
\begin{enumerate}
    \item $f: \bbR_{\text{Euclidean}} \to \bbR_{\text{Discrete}}$ is always an open map because take an open set $U \subset \bbR_{\text{Euclidean}}$, then we claim its image $f(U) \subset \bbR_{\text{Discrete}}$ is open - to check, take ball of $\epsilon = 0.5$ around any point $y\in f(U)$, then the only point in $B_\epsilon(y)$ is $y$ itself, hence $f(U)$ is open.

    \item $f: \bbR \to \{0,1\}$ such that $f(x)=0$ for $x\leq 0$ and $f(x)=1$ for $x>0$. Clearly discontinuous (jump discontinuity) but open sets map to open sets since the image can only ever be $\emptyset, \{0\}, \{1\}$ or $\{0,1\}$ all of which are open in the metric space $\{0,1\}$. 

    \item $f:\bbR_{\text{Euclidean}} \to \bbZ_{\text{Discrete}}$ defined as $f(x) = \lfloor x \rfloor$. 

    \item Harmonic
    
\end{enumerate}
\end{proof}
\end{theorem}

\begin{theorem}
Compact to compact but not continuous.    
\begin{proof}
    Jump discontinuity $\bbR \to \bbR$.
\end{proof}
\end{theorem}

\begin{theorem}
    Connected to connected but not continuous.
    \begin{proof}
Topologist's sine curve.

$S^1 \to \bbR$ or $S^1 \to [0,2\pi)$.
\end{proof}
\end{theorem}

\begin{theorem}
Connected but not path connected.
\begin{proof}
Topologist's sine curve. Formally defined as $X \cup Y$ where $X=\{(x,\sin \frac{1}{x})\in \bbR^2: 0<x\leq \frac{1}{\pi}\}$ and $Y=\{(0,y) \in \bbR^2: -1\leq y\leq 1\}$.
\end{proof}
\end{theorem}

\begin{theorem}
    Complete and bounded but not compact.
    \begin{proof}
        (0,1)
    \end{proof}
\end{theorem}

\begin{theorem}
Continuous bijection that is not a homeomorphism (i.e., whose inverse is not continuous).

\begin{proof}
$f: [0,2\pi) \to S^1$ defined as $f(x)=(\cos x, \sin x)$ since for the point $(1,0)\in S^1$, the inverse evaluates to $0$ but approaching from below, $f^{-1}(2\pi -\epsilon)$ approaches $2\pi$. 
\end{proof}
\end{theorem}

\begin{theorem}
\begin{enumerate}
    \item  Closed and totally bounded set (not complete) that is not compact.
    \item Closed and complete set (not totally bounded) that is not compact.
\end{enumerate}

\begin{proof}
    \begin{enumerate}
        \item $[0,1] \cap \bbQ$.
        \item Closed unit ball in infinite dimensions.
    \end{enumerate}
\end{proof}
\end{theorem}

\bigskip

\section*{Proofs}

\begin{theorem}
    Cantor set properties: compact, empty interior, nowhere dense, uncountable, totally disconnected.
\end{theorem}

\begin{theorem}
    If every closed and bounded subset of a metric space $M$ is compact, then $M$ is complete.
\end{theorem}

\begin{theorem}
    If $M$ is connected and $f : M \to N$ is continuous then $f(M)$ is connected. In particular, if $f$ is surjective then $N$ is connected. 
\end{theorem}

\begin{theorem}
    If a sequence in $\bbR^n$ converges on one metric then it converges on the other metrics.  
\end{theorem}

\begin{theorem}
    Closed and totally bounded and complete implies compact.
\end{theorem}

\begin{theorem}
A countable intersection of nested non-empty compact sets is compact and non-empty.
\end{theorem}

\begin{theorem}
    Cantor set is compact.
\end{theorem}

\begin{theorem}
    Path connected implies connected.
\end{theorem}

\begin{theorem}
    Intersection of compact sets is compact.
    \begin{proof}
        Let $(K_i)$ be a collection of compact sets. Then, we want to show that $\bigcap_i K_i$ is compact.

        Let $m_1, m_2, \dots$ be a sequence in $\bigcap_i K_i$. Clearly, all the $m_j$'s lie in each of the $K_i$'s (since it is in the intersection), so in particular all the $m_j$'s lie in $K_1$, which is compact, so there is a convergent subsequence $(m_{j_n})$ which converges to $m^* \in K_1$. Let us show that $m^* \in K_q$ for arbitrary $q$.

        Since $(m_{j_n})$ is a subsequence of $(m_j)$ and all the $m_j$'s lie in each of the $K_i$'s, then clearly $(m_{j_n})$ must be in $K_q$. So $(m_{j_n})$ converges to $m^* \in K_q$. 

        Hence $m^*\in \bigcap_i K_i$.
    \end{proof}
\end{theorem}

\begin{theorem}
    Continuous function defined on a compact metric space $M$ is uniformly continuous.
\end{theorem}

\begin{theorem}
    Continuous bijection defined on a compact domain is a homeomorphism (i.e., its inverse is automatically continuous).
\end{theorem}

\begin{theorem}
Every connected subset of $\bbR$ is path connected. Every connected open subset of $\bbR^n$ is path connected.
\end{theorem}

\end{document}
