\documentclass[11pt]{article}

%----------Packages----------
\usepackage{color}
\usepackage{amsmath}
\usepackage{amssymb}
\usepackage{amsthm}
\usepackage{amsrefs}
\usepackage{mathrsfs}
\usepackage{mathtools}
\usepackage[mathcal]{eucal}
\usepackage{enumerate}
\usepackage[shortlabels]{enumitem}
\usepackage{verbatim}
\usepackage{fullpage}
\usepackage{multicol}

%----------Commands----------
\clubpenalty=9999
\widowpenalty=9999

\newcommand{\bbF}{\mathbb{F}}
\newcommand{\bbN}{\mathbb{N}}
\newcommand{\bbQ}{\mathbb{Q}}
\newcommand{\bbR}{\mathbb{R}}
\newcommand{\bbZ}{\mathbb{Z}}

\renewcommand{\phi}{\varphi}
\renewcommand{\emptyset}{\O}

\renewcommand{\_}[1]{\underline{ #1 }}
\DeclarePairedDelimiter{\abs}{\lvert}{\rvert}
\DeclarePairedDelimiter{\norm}{\lVert}{\rVert}

\newcommand{\arr}{\longrightarrow}

\DeclareMathOperator{\ext}{ext}
\DeclareMathOperator{\bridge}{bridge}

%---------- Proof QED change ----------
\renewcommand{\qedsymbol}{QED}

%----------Theorems----------
\newtheorem{theorem}{Theorem}[section]
\newtheorem*{theorem*}{Theorem}
\newtheorem{proposition}[theorem]{Proposition}
\newtheorem{lemma}[theorem]{Lemma}
\newtheorem{corollary}[theorem]{Corollary}

\theoremstyle{definition}
\newtheorem{definition}[theorem]{Definition}
\newtheorem{remark}[theorem]{Remark}
\newtheorem{question}[theorem]{Question}

\numberwithin{equation}{subsection}

%----------Document----------
\begin{document}

\begin{center}
    {\LARGE\bfseries MATH 20700 Honors Analysis I}\\[6pt]
    {\Large Homework 2}\\[6pt]
    {\Large\itshape Malhar Manek}
\end{center}

\bigskip

\section*{Problem 5}
\begin{proof}
We can represent any point $z \in S^1$ by its angle $\theta_z = \angle(z) \in [0, 2\pi)$. Let $p, q, s$ be points in $S^1$ with corresponding angles $\theta_p, \theta_q, \theta_s$.

The distance is given by $d_a(p,q)=\min \{|\theta_p-\theta_q|, 2\pi - |\theta_p-\theta_q|\}$. Geometrically, this is the length of the shorter arc between $p$ and $q$ on the unit circle.

We check the four properties of a metric.

\begin{enumerate}
    \item We want to show that $d_a(p,q)\geq 0$. This is obvious since angles are in $[0, 2\pi)$, so $0\leq |\theta_p-\theta_q|< 2\pi$. Both $|\theta_p-\theta_q|$ and $2\pi - |\theta_p-\theta_q|$ are non-negative, so their minimum must be non-negative.

    \item We want to show that $d_a(p,q)=0$ if and only if $p=q$.
    Suppose $p=q$. Then $\theta_p = \theta_q$, so $|\theta_p-\theta_q|=0$. Thus, $d_a(p,q) = \min\{0, 2\pi\}=0$.
    Now suppose $d_a(p,q)=0$. This means $\min\{|\theta_p-\theta_q|, 2\pi - |\theta_p-\theta_q|\} = 0$. Since $|\theta_p-\theta_q| < 2\pi$, the term $2\pi - |\theta_p-\theta_q|$ must be positive. Therefore, we must have $|\theta_p-\theta_q|=0$, which implies $\theta_p = \theta_q$, and so $p=q$.

    \item We want to show that $d_a(p,q)=d_a(q,p)$. This is obvious since $|\theta_p-\theta_q|=|\theta_q-\theta_p|$.

    \item We want to show that $d_a(p,q) \leq d_a(p,s)+d_a(s,q)$.
    The distance $d_a(p,q)$ is the length of the shortest path along the circle's circumference from $p$ to $q$. The sum $d_a(p,s) + d_a(s,q)$ is the length of a particular path from $p$ to $q$: the one which goes from $p$ to $s$ (via the shorter arc) and then from $s$ to $q$ (again via the shorter arc). Since $d_a(p,q)$ is the length of the shortest possible path, it must be less than or equal to the length of this other path that goes via $s$. Thus, the inequality holds.

\end{enumerate}
Since all four properties hold, $d_a$ is a metric on $S^1$.
\end{proof}

\bigskip

\section*{Problem 6}
\begin{proof}
\begin{enumerate}
    \item Clearly, $\forall p,q\in [0,\frac{\pi}{2}]$, we have $|p-q| \in [0, \frac{\pi}{2}]$. On this interval, $\sin(x) \ge 0$, so $d_s(p,q) \ge 0$. Equality holds, $d_s(p,q) = \sin|p-q|=0$, if and only if $|p-q|=0$, i.e., $p=q$.
    
    \item Clearly, $\forall p,q\in [0,\frac{\pi}{2}]$, the symmetry of the absolute value function gives $d_s(p,q) = \sin|p-q| = \sin|q-p| = d_s(q,p)$.
    
    \item We claim that $\forall p,q,r\in [0,\frac{\pi}{2}]$, $d_s(p,q) \leq d_s(p,r)+d_s(r,q)$. This is equivalent to showing $\sin|p-q| \leq \sin|p-r| + \sin|r-q|$.
    
    By the triangle inequality for real numbers, we know $|p-q| \le |p-r| + |r-q|$. Since $\sin x$ is an increasing function on $[0, \frac{\pi}{2}]$, this implies:
    $$ \sin|p-q| \le \sin(|p-r| + |r-q|) $$
    Our task reduces to proving that $\sin(x+y) \le \sin x + \sin y$ for any $x, y \ge 0$ such that $x+y \in [0, \frac{\pi}{2}]$.
    
    Invoking calculus, let's fix $y \ge 0$ and define a function $f(t) = (\sin t + \sin y) - \sin(t+y)$ for $t \ge 0$. We note that the derivative of $\sin$ is $\cos$. The derivative of our function is:
    $$ f'(t) = \cos t - \cos(t+y) $$
    As $\cos$ is strictly decreasing on the interval $[0,\frac{\pi}{2}]$, and $t < t+y$, it follows that $\cos t > \cos(t+y)$, which means $f'(t) > 0$.
    
    Since its derivative is positive, $f(t)$ is an increasing function. Therefore, for any $x \ge 0$, we must have $f(x) \ge f(0)$. We calculate $f(0) = \sin 0 + \sin y - \sin y = 0$.
    
    Thus, $f(x) \ge 0$, which implies $\sin x + \sin y - \sin(x+y) \ge 0$, or $\sin(x+y) \le \sin x + \sin y$.
    
    Combining our results, we have:
    $$ d_s(p,q) = \sin|p-q| \le \sin(|p-r| + |r-q|) \le \sin|p-r| + \sin|r-q| = d_s(p,r)+d_s(r,q) $$
    The triangle inequality holds.
\end{enumerate}
Thus, $d_s$ is a metric.
\end{proof}

\bigskip

\section*{Problem 7}
\begin{proof}
Let $(p_n)$ be a convergent sequence in the metric space $M$. Then choosing $\epsilon =1$, we know that $\exists N \in \mathbb{N}$ such that $\forall n \geq N, |p_n-q|<1$. There are finitely many $ n\in \mathbb{N}$ such that $n<N$, so let $s=\max\{|p_n-q| : n<N\}$. Then, taking $r=\max\{s,1\}$ completes the proof.

\end{proof}

\section*{Problem 9}
\begin{proof}
Let us call the property that every bounded monotone sequence converges the ``monotone bounded sequence property”.

Since we know that $\bbR$ has the least upper bound property, it suffices to show that for a general metric space $M$, the least upper bound property implies monotone bounded sequence property.

Suppose every non-empty subset of the metric space $M$ that is bounded above has a supremum in $M$ (i.e., $M$ has the least upper bound property). Let $(x_n)$ be a monotone bounded sequence of points of $M$ and without loss of generality suppose $(x_n)$ is increasing (if it is decreasing then the proof is analogous with greatest lower bound property instead of least upper bound property). 

The set of points of this sequence $\{x_n\}$ is bounded and non-empty, hence it has a supremum $y\in M$. We want to show that ${x_n} \displaystyle \rightarrow y$. 

Let $\epsilon>0$ be arbitrary, then we want to show that $\exists N_\epsilon \in \bbN$ such that $\forall n \geq N_\epsilon, |x_n-y|<\epsilon$. 

Since the sequence is increasing, we know that if $n>m$ then $y\geq x_n\geq x_m$. Hence, it suffices to show that $\exists n_\epsilon$ such that $|x_{n_\epsilon}-y|<\epsilon$. Suppose for sake of contradiction that there does not exist such $n_\epsilon$. Then, $\forall n\in \bbN, |x_n-y|\geq \epsilon$, i.e., $x_n \leq y-\epsilon$ or $x_n\geq y+\epsilon$. Since $y$ is the supremum, the latter is ruled out, and we have that $\forall n\in \bbN, x_n\leq y-\epsilon\leq y$ giving a smaller upper bound and contradicting the assumption that $y$ is the supremum.

This proves that every monotone bounded sequence of points in $\bbR$ converges in $\bbR$. 

\end{proof}

\section*{Problem 11}
\begin{proof}
% Your proof here
\begin{enumerate}
    \item[a)] Let $(a_n)$ be a sequence of real numbers. We want to show that there exists either an increasing subsequence or a decreasing subsequence or both.

Case 1: An infinite decreasing subsequence exists.  Then we are done.

Case 2: No infinite decreasing subsequence exists.  
Then every decreasing subsequence is finite.  
In this case, we claim that there exists an increasing subsequence. 

For each index $i \in \mathbb{N}$ of the original sequence, define
\[
  d(i)
  =
  \max\bigl\{\,\ell\in\mathbb{N} \mid
    \exists\,i=n_1<n_2<\cdots<n_\ell\text{ with }a_{n_1}\ge a_{n_2}\ge\cdots\ge a_{n_\ell}
  \bigr\}.
\]
In words, $d(i)$ is the length of the longest finite decreasing subsequence that starts with the element at index $i$.
(The maximum exists because by hypothesis every decreasing subsequence is finite.)

Since each $d(i)$ is finite, for each $i$ we can choose a particular sequence of indices
\[
  s_{i,1},\;s_{i,2},\;\dots,\;s_{i,d(i)},\qquad
  s_{i,1}=i,\;\;s_{i,1}<s_{i,2}<\dots<s_{i,d(i)},
\]
with 
\(
  a_{s_{i,j}}\ge a_{s_{i,k}}\text{ whenever }j<k,
\)
and set  $L(i)=s_{i,d(i)}$
as the last index of this chosen subsequence.

Now define inductively
\[
  n_1 = L(1),
  \qquad
  n_{t+1} = L\bigl(n_t + 1\bigr)
  \quad (t\ge 1).
\]

We claim that the subsequence $\bigl(a_{n_t}\bigr)_{t\ge1}$ is strictly increasing.

Since $L(i)$ is the last term of a maximal decreasing subsequence that starts at $i$, we cannot extend that subsequence any further.
Hence  
\[
  j > L(i)
  \;\;\Longrightarrow\;\;
  a_{L(i)} < a_j.
\]

By construction, $n_{t+1}=L(n_t+1)\ge n_t+1>n_t$, so $n_1<n_2<\dots$.

Take $i = n_t+1$ and set $j=n_{t+1}$, then
\[
  a_{n_t} = a_{L(n_t+1)} < a_{n_{t+1}}\text{ for every }t.
\]
Thus $a_{n_t}<a_{n_{t+1}}$, proving the claim.

\medskip
\noindent
Consequently, in Case~2 we get an infinite increasing subsequence, while in Case~1 an infinite decreasing subsequence already exists.  
Therefore every sequence of real numbers has an increasing subsequence or a decreasing subsequence (or both).

This completes the proof.

\item[b)] Part a) tells us that every sequence in $\bbR$ has a monotone subsequence. In particular, every bounded sequence in $\bbR$ has a bounded monotone subsequence, which, by Problem 9, is a convergent subsequence.

\item[c)] Bolzano-Weierstrass theorem says that every bounded sequence has a convergent subsequence. By part b), we have an alternative proof of this.

\item[d)] Heine-Borel theorem says that every closed and bounded subset of $\bbR^n$ is compact (though in this case we are concerned only with closed and bounded subsets of $\bbR$). 

By the above, we know that every bounded sequence has a convergent subsequence. Hence, every sequence in a closed and bounded set has a subsequence that converges to a point in the set, and this is exactly the definition of compactness.

Hence, Heine-Borel theorem generalises the Bolzano-Weierstrass theorem. 
    
\end{enumerate}
\end{proof}

\section*{Problem 12}
\begin{proof}
\begin{enumerate}
    \item[a)] Yes, limits are indeed unaffected by rearrangement under a bijective function. Suppose the sequence $(p_n)$ converges to $x$ and fix some arbitrary $\epsilon>0$.

    By definition of convergence, $\exists N^* \in \bbN$ such that for all $n\geq N^*, |p_n-x|<\epsilon$. Since there are finitely many indices $i<N^*$, write them as $A =\{i \mid i <N^*, 1\leq i\leq j\}$, then compute the set \[
    B = \{f^{-1}(i) \mid i \in A\}.
    \]

Clearly, $B$ is finite (since $A$ is finite and the function $f$ is bijective) so take $b^*=\max(B)$. Remember that by construction, $\forall b\in B, |p_{f(b)}-x|\geq \epsilon$. Then, since $b^*$ is the maximum, we clearly have $\forall n\geq b^*, |q_n-x|<\epsilon$. This proves that $(q_n)$ converges to $x$. 

\item[b)] If the limit of $p_n$ exists and $f$ is injective, we repeat the exact same argument as in part a). Again, the construction is well-defined because $f^{-1}$ is well-defined when $f$ is injective.

However, if the limit of $p_n$ does not exist, then it may be possible that the rearrangement under an injective function does have a limit.

Consider the sequence $p_n=-1,1,-1,1,-1,1,\dots$ which clearly diverges. The injective map defined by $\forall i, f(i)=2i$ produces the sequence $q_n =1,1,1,1,\dots$ which clearly converges to $1$.

\item [c)] Suppose $f$ is surjective but not injective. We claim that limits may not be preserved under rearrangement in this case. 

Consider the following counterexample. Define the sequence $(p_n)$ as $p_1=-1$ and $\forall i\geq 2, p_i=1$. Clearly, this sequence converges to $1$. 

Now consider the surjective function $f$ defined as $f(n)=1$ if $n$ is odd and $f(n) = \frac{n}{2}+1$ if $n$ is even. 

Under the rearrangement defined by this function, we have $q_n=-1,1,-1,1,-1,1,\dots$ and it is easy to see that $q_n$ is not Cauchy (because $\forall j\geq 2, |q_j-q_{j-1}|=2$), hence it is not convergent.

\end{enumerate}

\end{proof}

\section*{Problem 13}
\begin{proof}
We construct the sequence $m_n$ such that $\forall n, m_{2n}=p_n, m_{2n-1}=p$. Clearly, $m_n$ converges to $p$.

By the assumption, the sequence $f(m_n)$ converges. Moreover, the subsequence $f(m_{2n-1})=f(p)$ is a constant and converges to $f(p)$. 

Thus, we have a convergent sequence $f(m_n)$ which has a subsequence $f(m_{2n-1})$ which converges to $f(p)$, hence we know that the sequence itself, and any other subsequences, must all converge to the same point $f(p)$.

Hence, we conclude that the sequence $f(m_{2n})=f(p_n)$ converges to $f(p)$. We have hereby proved that $f$ maps the convergent sequence $p_n \to p$ to the convergent sequence $f(p_n) \to f(p)$. This is exactly the definition of a continuous function.

\end{proof}

\section*{Problem 17}
\begin{proof}
The letters C, G, I, J, L, M, N, O, S, U, V, W, Z are homeomorphic to each other, because each of them can be unbent into a single straight line.

D, P and O are homeomorphic to each other because they can each be unbent into a circle.

E, F, T and Y are homeomorphic to each other because in a sense each of them have one `junction'.

K and X are homeomorphic to each other because in a sense each of them has two `junctions'.

A, Q, R, H and B are not homeomorphic to any letter.

None of them are isometric to any other, because the purpose of the alphabet is to make letters distinctly recognisable, ensuring that no two letters are isometric.

\end{proof}

\section*{Problem 22}
\begin{proof}
Yes. To prove this, consider a Cauchy sequence $(x_n)$ in $M$ and let $X=\{x_n \mid n\in \bbN\} \subset M$. We claim that is $X$ is bounded (Claim 1). 

Consider the closure of $X$, denoted $\overline{X}$, which is clearly closed. We claim that the boundedness of $X$ implies that $\overline{X}$ is bounded (Claim 2).

Thus, $\overline{X}$ is closed and bounded, hence by assumption, it is compact. Hence, every sequence of points in $X$ has a convergent subsequence. In particular, the original Cauchy sequence $(x_n)$ has a convergent subsequence. 

We claim that if a Cauchy sequence has a subsequence that converges to $y$, then the original Cauchy sequence itself also converges to $y$ (Claim 3). Then, we see that every Cauchy sequence in $M$ converges, showing that $M$ is complete.

Let us now prove our claims.

\textbf{Claim 1}:  Let $(x_n)$ be a Cauchy sequence in a metric space $M$. By definition, for $\epsilon = 1$, there exists an integer $N$ such that for all $m, n \geq N$, $d(x_m, x_n) < 1$.
In particular, for all $n \geq N$, we have $d(x_n, x_N) < 1$. This means all terms from $x_N$ onwards are contained in the ball $B_1(x_N)$.

Now consider the finite set of the first $N-1$ terms: $\{x_1, x_2, \ldots, x_{N-1}\}$. Let $R$ be the maximum distance from these points to $x_N$, plus the radius $1$ we are already considering:
$$ R = \max\{d(x_1, x_N), d(x_2, x_N), \ldots, d(x_{N-1}, x_N), 1\} $$
For any term $x_k$ in the sequence, we see that if $k < N$, then $d(x_k, x_N) \leq R$ by definition of $R$. And if $k \geq N$, then $d(x_k, x_N) < 1 \leq R$.

Therefore, for all $k \in \mathbb{N}$, $d(x_k, x_N) \leq R$. This shows that the entire set of points $X = \{x_n \mid n \in \mathbb{N}\}$ is contained within the closed ball $\overline{B}_R(x_N)$, and is therefore bounded. 

\textbf{Claim 2}: Let $X$ be a bounded set. By definition, there exists a point $p \in M$ and a radius $R > 0$ such that for all $x \in X$, $d(p, x) < R$. This means $X \subseteq B_R(p)$.

We want to show that $\overline{X}$ is also bounded. Let $y$ be an arbitrary point in $\overline{X}$. By the definition of the closure, any open ball centered at $y$ must contain a point from $X$. Let's consider the ball $B_1(y)$. There must exist a point $x_0 \in X$ such that $x_0 \in B_1(y)$, which implies $d(y, x_0) < 1$.

Since $x_0 \in X$, we know that $d(p, x_0) < R$.
By the triangle inequality, we can bound the distance $d(p, y)$:
$$ d(p, y) \leq d(p, x_0) + d(x_0, y) < R + 1 $$
Since $y$ was an arbitrary point in $\overline{X}$, this inequality holds for all points in $\overline{X}$. Thus, $\overline{X}$ is contained within the ball $B_{R+1}(p)$, which proves that $\overline{X}$ is bounded.

\textbf{Claim 3}:  Let $(x_n)$ be a Cauchy sequence and let $(x_{n_k})$ be a subsequence that converges to $y$, i.e., $\lim_{k \to \infty} x_{n_k} = y$. We want to show that $\lim_{n \to \infty} x_n = y$.

Let $\epsilon > 0$ be arbitrary. 

Since $(x_n)$ is a Cauchy sequence, there exists an integer $N_1$ such that for all $m, n > N_1$, we have $d(x_m, x_n) < \frac{\epsilon}{2}$.

Since $x_{n_k} \to y$, there exists an integer $K$ such that for all $k > K$, we have $d(x_{n_k}, y) < \frac{\epsilon}{2}$.

We need to find an $N$ such that for all $n > N$, $d(x_n, y) < \epsilon$. Choose an index $k'$ from the subsequence large enough so that both $k' > K$ and the corresponding term index $n_{k'} > N_1$. We can always do this because $n_k \to \infty$ as $k \to \infty$.

Let $N = n_{k'}$. Now consider any $n > N$.
Since $n > N = n_{k'}$ and $n_{k'} > N_1$, both $n$ and $n_{k'}$ are greater than $N_1$. Therefore,
$$ d(x_n, x_{n_{k'}}) < \frac{\epsilon}{2} $$
Also, since $k' > K$,
$$ d(x_{n_{k'}}, y) < \frac{\epsilon}{2}  $$
Using the triangle inequality:
$$ d(x_n, y) \leq d(x_n, x_{n_{k'}}) + d(x_{n_{k'}}, y) < \frac{\epsilon}{2} + \frac{\epsilon}{2} = \epsilon $$
This holds for all $n > N$, so by definition, the sequence $(x_n)$ converges to $y$.

*** 

Unrelated side remarks: The infinite dimensional unit sphere (as seen in lecture) is closed and bounded but not compact. However, it is complete because there are no Cauchy sequences!




\end{proof}

\section*{Problem 28}
\begin{proof}
\begin{enumerate}
    \item[a)] Not necessarily. For a counterexample, consider $M= \bbR$ with the Euclidean metric and $N= \bbR$ with the discrete metric.

    In this case, for \textbf{any} function $f: M \to N$, if $U\subseteq M$ is open then $f(U) \subseteq N$ is open. To see that $f(U)$ is indeed open, pick any arbitrary point $x\in f(U)$ and we want to show that $\exists \epsilon>0$ such that $B_\epsilon(x) \subset U$. In particular, $\epsilon = 0.21$ works for any such $x$, because by the definition of the discrete metric, the only point in $B_{0.21}(x)$ is $x$ itself! Any other point is a distance 1 away from $x$. Then clearly $\forall x\in f(U), B_{0.21}(x) \subset f(U)$. 

    In this argument, we never assumed anything about the function $f$ itself, so it could be either continuous or discontinuous and the argument would remain the same. 
    

    \item[b)] Yes. We know that continuous functions satisfy the property that the preimage of an open set is open. 

    Since $f$ is a homeomorphism, we know that $f^{-1}: N \to M$ is continuous, hence given an open set $U \subseteq M$ the preimage under $f^{-1}$ is open, hence $(f^{-1})^{-1}(U)=f(U)$ is open.

    \item[c)] Yes. Because $f$ is a bijection, we know that $f^{-1}:N \to M$ is a bijection, hence it suffices to show that $f^{-1}$ is continuous. Let $U \subseteq M$ be open, then its preimage under the inverse is $(f^{-1})^{-1}(U)=f(U)$ which is open since $f$ is an open map.

    \item[d)] No. Consider the function $f: \bbR \to \bbR $ defined as $\forall x<0, f(x)=x$, $\forall 0\leq x\leq 1, f(x)=0, \forall x>1, f(x)=x-1$. 

    Then take the open interval $U=(0.21, 0.73) \subset \bbR$, then its image is $f(U)=\{0\}$ which is not open.  

    \item[e)] Yes. We know $f$ and $f^{-1}$ are continuous and $f$ is surjective, hence it suffices to show that $f$ must be injective. In particular, it suffices to show that $f$ must be monotone.

    Suppose for sake of contradiction that $f: \bbR \to \bbR$ is not monotone, then it must have a local maximum or a local minimum (or both). 

    Without loss of generality, assume that $f$ has a local maximum at $x \in \bbR$, i.e., $\exists \epsilon>0$ such that $\forall y \in (x-\epsilon, x+\epsilon), f(y)<f(x)$. Write $X=(x-\epsilon, x+\epsilon)$, then it is easy to see that $X$ is open, and we claim that its image $f(X)$ is not open, contradicting the assumption that $f$ is an open map.

    To prove the claim, consider $f(x) \in f(X)$, then clearly $\not \exists \delta>0$ such that $(f(x)-\delta, f(x)+\delta) \subset f(X)$, because $f(x)$ is the maximum point of $f(X)$ so any $f(x)+\delta$ is no longer in $f(X)$ for $\delta>0$. 
    

    \item[f)] No. We claim for a counterexample that there exists a continuous open surjection $f: S^1 \to S^1$ that is not a homeomorphism.

    For any point on the unit circle $S^1$, the point can be fully characterised by its angle $\theta$ with respect to the positive $x$-axis. Then, we define $\forall \theta \in [0,2\pi), f(\theta)=2\theta \mod 2\pi$. 

    Clearly $f$ is continuous. The inverse is also simply $f^{-1}(\theta)=\frac{\theta}{2}$ hence it is continuous, and $f$ is open. It is also clearly surjective since $\forall \theta \in [0,2\pi)$, we have $f(\frac{\theta}{2})=\theta$.

    But this is not a bijection, since $f(0)=f(\pi)=0$, hence it is not injective.
    

    
    
\end{enumerate}

\end{proof}

\section*{Problem 101}
\begin{proof}
We know that $\displaystyle \sum_{i=1}^\infty \frac{1}{2^i} = 1$, hence $\forall a,b \in \sum, 0\leq d(a,b) \leq 1$. 


\begin{enumerate}
    \item[a)] We want to show that for any sequence $(x_1,x_2,\dots)$ with $x_1, x_2, \dots \in \sum$, we can find a convergent subsequence.

Since this is a sequence of sequences, let us use the notation $x_{ij}$ such that $x_{12}$, for example, is the 2nd element of $x_1$.

Let us construct the convergent subsequence inductively.

For $j=1$, there must be infinitely many sequences in our original sequence such that their first component $x_{i1}$ is the same (either 1 or 0).

Then, we can repeat this process. From the infinite subsequence we just found, we can find an infinite sub-subsequence where the second component $x_{i2}$ is the same.

Continuing this process, we construct a subsequence and a limit point. Let the original sequence be $S_0 = (x_1, x_2, x_3, \dots)$.

First, we construct the limit point, which we will call $y = (y_1, y_2, \dots)$.
\begin{enumerate}
    \item Consider the sequence of first components $(x_{11}, x_{21}, x_{31}, \dots)$. As this is a sequence of 0s and 1s, one value must appear infinitely often. Let this value be $y_1$. Let $S_1$ be the subsequence of $S_0$ consisting of all sequences whose first component is $y_1$.
    \item Now consider the second components of the sequences in the infinite subsequence $S_1$. Again, one value must appear infinitely often. Let this be $y_2$. Let $S_2$ be the subsequence of $S_1$ whose sequences all have $y_2$ as their second component.
    \item We continue this process inductively. Having constructed the infinite subsequence $S_{k-1}$, we define $y_k$ to be the value that appears infinitely often in the $k$-th position of sequences in $S_{k-1}$, and we define $S_k$ as the subsequence of $S_{k-1}$ where all sequences share $y_k$ as their $k$-th component.
\end{enumerate}
This procedure gives us a candidate for the limit, the sequence $y = (y_1, y_2, y_3, \dots) \in \sum$.

Next, we construct the convergent subsequence, let's call it $(z_k)_{k \ge 1}$, by taking the ``diagonal" of our subsequences. Let $z_1$ be the first sequence in $S_1$. Let $z_2$ be the second sequence in $S_2$. In general, let $z_k$ be the $k$-th sequence in $S_k$.

By this construction, the sequence $z_k$ is an element of $S_j$ for all $j \le k$. This means the first $k$ components of $z_k$ are precisely $(y_1, y_2, \dots, y_k)$.

To see that $(z_k)$ converges to $y$, fix any arbitrary $\epsilon>0$. Choose any $m\in \mathbb{N}$ such that $\frac{1}{2^{m-1}}<\epsilon$. For any index $k \ge m$ in our subsequence, we see that the distance between $z_k$ and $y$ is:
$$d(z_k, y) = \sum_{n=1}^\infty \frac{|z_{kn} - y_n|}{2^n}$$
where $z_{kn}$ is the $n$-th component of the sequence $z_k$. By our construction, the first $k$ components of $z_k$ match the first $k$ components of $y$. Therefore, $z_{kn} = y_n$ for all $1 \le n \le k$. This means the first $k$ terms in the sum for the distance are zero.
$$d(z_k, y) = \sum_{n=k+1}^\infty \frac{|z_{kn} - y_n|}{2^n}$$
Since the components are either 0 or 1, the numerator $|z_{kn} - y_n|$ is at most 1. We can therefore bound the sum:
$$d(z_k, y) \le \sum_{n=k+1}^\infty \frac{1}{2^n}$$
This is a geometric series which sums to $\frac{1}{2^k}$. Since we are considering $k \ge m$, we have:
$$d(z_k, y) \le \frac{1}{2^k} \le \frac{1}{2^m} < \frac{1}{2^{m-1}} < \epsilon$$
Thus, for any $\epsilon > 0$, we have found an $m$ such that for all $k \ge m$, $d(z_k, y) < \epsilon$. This proves that the subsequence $(z_k)$ converges to $y$.

Thus, every sequence has a convergent subsequence, hence $\sum$ is compact.
    
    
    \item[b)] As we saw in lecture, every element of the Cantor set can be written as a decimal expansion with infinite string/sequence of 0's and 2's. (In the $n$-th iteration of the middle-thirds construction of the Cantor set, if the point lies right of the middle then set the $n$-th term of the sequence as $2$, and if the point lies left of the middle then set the $n$-th term of the sequence as $0$). 

    Then there is a natural bijection to construct from $\sum$ to the Cantor set: take any element (string of 0's and 1's) in $\sum$, then map it to the same element in the Cantor set, with 0 corresponding to 0 and 1 corresponding to 2. 


    Formally, let the Cantor set $C$ be the set of all real numbers in $[0, 1]$ that admit a ternary expansion using only the digits 0 and 2. That is,
$$C = \left\{ x \in [0, 1] \;\middle|\; x = \sum_{n=1}^{\infty} \frac{c_n}{3^n} \text{ where } c_n \in \{0, 2\} \text{ for all } n \ge 1 \right\}.$$
Let $\sum$ be the set of all infinite sequences of zeroes and ones.

We define the function $f: \sum \to C$ as follows. For any sequence $a = (a_n)_{n=1}^{\infty} = (a_1, a_2, a_3, \dots) \in \sum$, we map it to the point $f(a) \in C$ given by
$$f(a) = \sum_{n=1}^{\infty} \frac{2a_n}{3^n}.$$

Now we want to check that $f$ is bijective, continuous and its inverse is also continuous. We know that $\sum$ is compact, hence it suffices to check that $f$ is bijective and continuous. (There is a theorem which says that a continuous bijection defined on a compact domain is a homeomorphism).

\textbf{Injectivity:}
Let $a = (a_n)$ and $b = (b_n)$ be two distinct sequences in $\sum$. Since $a \neq b$, there must be a first index $k \ge 1$ at which they differ. Let's assume without loss of generality that $a_k = 1$ and $b_k = 0$, while $a_n = b_n$ for all $n < k$. We want to show that $f(a) \neq f(b)$.

Consider the difference $f(a) - f(b)$:
$$ f(a) - f(b) = \sum_{n=1}^{\infty} \frac{2(a_n - b_n)}{3^n} = \frac{2(a_k - b_k)}{3^k} + \sum_{n=k+1}^{\infty} \frac{2(a_n - b_n)}{3^n} $$
Since $a_n = b_n$ for $n < k$, those terms are zero. As $a_k=1$ and $b_k=0$:
$$ f(a) - f(b) = \frac{2}{3^k} + \sum_{n=k+1}^{\infty} \frac{2(a_n - b_n)}{3^n} $$
The terms in the remaining sum are bounded, since $a_n, b_n \in \{0, 1\}$, which means $a_n - b_n \in \{-1, 0, 1\}$. The sum is bounded below by the case where $a_n=0$ and $b_n=1$ for all $n > k$:
$$ \sum_{n=k+1}^{\infty} \frac{2(a_n - b_n)}{3^n} \ge \sum_{n=k+1}^{\infty} \frac{-2}{3^n} = -2 \sum_{n=k+1}^{\infty} \left(\frac{1}{3}\right)^n $$
This is a geometric series with sum $-2 \left( \frac{1/3^{k+1}}{1 - 1/3} \right) = -2 \left( \frac{1/3^{k+1}}{2/3} \right) = -\frac{1}{3^k}$.
Therefore,
$$ f(a) - f(b) \ge \frac{2}{3^k} - \frac{1}{3^k} = \frac{1}{3^k} > 0 $$
This implies $f(a) > f(b)$, so $f(a) \neq f(b)$. Thus, $f$ is injective.

\textbf{Surjectivity:}
Let $x$ be an arbitrary element of the Cantor set $C$. By the definition of $C$, $x$ has a ternary expansion consisting only of the digits 0 and 2:
$$ x = \sum_{n=1}^{\infty} \frac{c_n}{3^n} \quad \text{where } c_n \in \{0, 2\} \text{ for all } n \ge 1 $$
We need to find a sequence $a = (a_n) \in \sum$ such that $f(a) = x$.
Let's construct the sequence $a$ by defining its terms as $a_n = c_n / 2$.
Since $c_n$ is either 0 or 2, $a_n$ is either $0/2=0$ or $2/2=1$. Therefore, $a = (a_n)$ is an infinite sequence of zeroes and ones, so $a \in \sum$.

Now we apply the function $f$ to this sequence $a$:
$$ f(a) = \sum_{n=1}^{\infty} \frac{2a_n}{3^n} = \sum_{n=1}^{\infty} \frac{2(c_n/2)}{3^n} = \sum_{n=1}^{\infty} \frac{c_n}{3^n} = x $$
Since for any $x \in C$ we have constructed a preimage $a \in \sum$, the function $f$ is surjective.

Since $f$ is both injective and surjective, it is bijective.

Now let us check that $f$ is continuous. Let $a,b\in \sum$. 

We begin by considering the distance between their images in the Cantor set $C$, which is the standard absolute value distance in $\mathbb{R}$:
$$|f(a) - f(b)| = \left| \sum_{n=1}^{\infty} \frac{2a_n}{3^n} - \sum_{n=1}^{\infty} \frac{2b_n}{3^n} \right| = \left| \sum_{n=1}^{\infty} \frac{2(a_n - b_n)}{3^n} \right|$$
By the triangle inequality for infinite series ($|\sum x_n| \le \sum |x_n|$), we can bound this expression:
$$|f(a) - f(b)| \le \sum_{n=1}^{\infty} \left| \frac{2(a_n - b_n)}{3^n} \right| = \sum_{n=1}^{\infty} \frac{2|a_n - b_n|}{3^n}$$
Now, we relate this expression to the metric $d(a,b) = \sum_{n=1}^\infty \frac{|a_n - b_n|}{2^n}$. For any integer $n \ge 1$, we know that $2^n < 3^n$, which implies $\frac{1}{3^n} < \frac{1}{2^n}$. Since $|a_n - b_n|$ is non-negative, we can write:
$$\frac{|a_n - b_n|}{3^n} \le \frac{|a_n - b_n|}{2^n}$$
Substituting this inequality back into our sum, we get:
$$\sum_{n=1}^{\infty} \frac{2|a_n - b_n|}{3^n} \le \sum_{n=1}^{\infty} \frac{2|a_n - b_n|}{2^n} = 2 \left( \sum_{n=1}^{\infty} \frac{|a_n - b_n|}{2^n} \right)$$
The expression in the parenthesis is precisely the definition of the metric $d(a,b)$. Therefore, we have established the relationship:
$$|f(a) - f(b)| \le 2 \, d(a,b)$$
Now we can show continuity using the standard $\epsilon$-$\delta$ definition. Let $\epsilon > 0$ be given. We must find a $\delta > 0$ such that if $d(a,b) < \delta$, then $|f(a) - f(b)| < \epsilon$.

Choose $\delta = \frac{\epsilon}{2}$.
If $d(a,b) < \delta$, then from our derived inequality:
$$|f(a) - f(b)| \le 2 \, d(a,b) < 2\delta = 2\left(\frac{\epsilon}{2}\right) = \epsilon$$
Thus, for any $\epsilon > 0$, we have found a $\delta > 0$ that satisfies the condition for continuity. This completes the proof.


    
\end{enumerate}

\end{proof}

\end{document}
